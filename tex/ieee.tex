\documentclass[lettersize,journal]{IEEEtran}
\usepackage{amsmath,amsfonts}
\usepackage{algorithmic}
\usepackage{algorithm}
\usepackage{array}
\usepackage[caption=false,font=normalsize,labelfont=sf,textfont=sf]{subfig}
\usepackage{textcomp}
\usepackage{stfloats}
\usepackage{url}
\usepackage{verbatim}
\usepackage{graphicx}
\usepackage{cite}

% added packages from authors
\usepackage{xcolor}
\usepackage{booktabs} % for \midrule in tables
\usepackage{hhline}
\usepackage{bbm}
\usepackage{bm}
\usepackage{makecell}
\usepackage[mode=buildnew]{standalone}
\usepackage{tikz}
\usetikzlibrary{calc,arrows,patterns,intersections}
\usepackage{pgfplots}
\usepackage{xcolor}
\usepgfplotslibrary{fillbetween}

\hyphenation{op-tical net-works semi-conduc-tor IEEE-Xplore}

\begin{document}


\title{A New Paradigm for Energy Intensive Industries: Adapting to Provide Power Flexibility}

\author{Peter A.V. Gade\textsuperscript{*}\textsuperscript{\textdagger}, Trygve Skjøtskift\textsuperscript{\textdagger}, Charalampos Ziras\textsuperscript{*}, Henrik W. Bindner\textsuperscript{*}, Jalal Kazempour\textsuperscript{*} \\
    \textsuperscript{*}Department of Wind and Energy Systems, Technical University of Denmark, Kgs. Lyngby, Denmark \\
    \textsuperscript{\textdagger}IBM Client Innovation Center, Copenhagen, Denmark
    % <-this % stops a space
    \thanks{Corresponding author. Tel.: +45 24263865. \\ Email addresses: pega@dtu.dk (P.A.V. Gade), Trygve.Skjotskift@ibm.com (T. Skjøtskift), hwbi@dtu.dk (H.W. Bindner), jalal@dtu.dk (J. Kazempour).}% <-this % stops a space
    \thanks{Manuscript received April 19, 2021; revised August 16, 2021.}}

% \footnote{Corresponding author. Tel.: +45 24263865. \\ Email addresses: pega@dtu.dk (P.A.V. Gade), Trygve.Skjotskift@ibm.com (T. Skjøtskift), hwbi@dtu.dk (H.W. Bindner), jalal@dtu.dk (J. Kazempour).}

% The paper headers
\markboth{Journal of \LaTeX\ Class Files,~Vol.~14, No.~8, August~2021}%
{Shell \MakeLowercase{\textit{et al.}}: A Sample Article Using IEEEtran.cls for IEEE Journals}

\IEEEpubid{0000--0000/00\$00.00~\copyright~2021 IEEE}
% Remember, if you use this you must call \IEEEpubidadjcol in the second
% column for its text to clear the IEEEpubid mark.

\maketitle

% \tableofcontents

\begin{abstract}
    % \input{sections/abstract}
    In order to reach carbon emission goals, current infrastructure can and should help balance the power grid. Single-state industry processes can be particularly suited for this purpose which is exemplified using a real-world case study of a zinc galvanizing process. It is shown how modest investments into the power control of the furnace enables flexible provision in two ancillary services, FCR and mFRR, with a pay-back time potentially as low as a year. The monetary value of both services are significant with FCR being the preferable service as its impact on the temperature of the zinc is negligible. This paper serves to illustrate how current inflexible demand can become flexible and profitable while aiding the green transition.
\end{abstract}

\begin{IEEEkeywords}
    Demand-side flexibility.
\end{IEEEkeywords}

\input{sections/introduction}

Power prices have increased massively in 2022, and the increasing share of renewable energy stresses the grid at to an unprecedented degree. In 2022, the Danish transmission operator (TSO), Energinet, spent 2.7 bil. on ancillary services, an increase of 1.3 billion from 2021 \cite{energinetOmkostninger}. Energinet expect this cost will only increase further as the green transition demands more renewable energy and a clean power grid with less to none CO$_{2}$ emissions.

To facilitate the green transition by using greener technologies to balance the power grid, existing infrastructure can readily be used. In particular, existing power demand can to some degree become flexible, i.e., the power consumption can vary within some thresholds for a certain amount of time as a response from an \textit{aggregator}. IBM has created a solution, Flex Platform, that harness demand-side flexibility to be bid into various ancillary service markets. Power demand in the industrial sector is significant and for simple, single-state industry processes, modest investments can enable it to become flexible.

Single-state processes are characterized by being solely sequential, i.e., the next step occurs after the previous step is done. They are repetitive and well, relatively simple and uncomplicated. The \textit{state} is often a temperature, e.g., zinc temperature in a galvanizing furnace as shown later. Such processes are prone to become flexible with respect to their power consumption as it required very little to none extra effort or adjustments to the process.

In the European Union and Denmark, there is an increased focus in on $CO_{2}$ \textit{accounting}. Companies, small and large, all need to look seriously into their current carbon emissions and especially initiatives that reduce them. Demand-side flexibility is an obvious choice for energy intensive single-state industry process as it allows them to balance the power grid\footnote{With the assumption that it would otherwise be done by fossil fuel power plants.}. Other initiatives include energy optimization and energy efficiency which can be done to a certain degree.

In this work, an industry process is investigated, as exemplified by a zinc galvanizing furnace using real data (see Section \ref{sec:zinc_furnace_description} for details). First, we describe the process in detail and show how the thermal characteristics of the zinc and furnace can be described. Second, we show how investments into a more granular, continuous control of the power supply to the furnace enables it to deliver power flexibility. Two quite different ancillary services are targeted: frequency containment reserves (FCR) and manual frequency restoration reserves (mFRR). For both, full-hindsight optimization are provided to compute upper-bound potentials for 2022. Lastly, an analysis of the zinc temperature's impact on the flexibility potential is carried out.

\subsection{Research questions}

In this work, we investigate how its viable for zinc furnace to adapt for provision of power flexibility through ancillary services. We do so from a techno-economic perspective while also investigating different ancillary services. The research questions we investigate are therefore:

\begin{itemize}
    \item Are there incentives for an energy intensive industry process position to deliver power flexibility to the grid?
    \item How can it adapt its industry process to provide power flexibility?
    \item What types of ancillary services are well suited to an energy intensive industry process?
    \item What is the monetary impact of the degrees of freedom (temperature thresholds) in the process?
\end{itemize}

The first question is answered by looking at the overall case study, combining carbon emission and monetary incentives. The second question is answered by thoroughly investigating the operational baseline power consumption profile of the zinc furnace, and giving recommendations on the investments needed to cater for power flexibility provision. The third question looks at the two revenue streams for the zinc furnace's flexibility potential: FCR with a low but fast energy delivery and mFRR with a high but slow energy delivery. The fourth question is answered by conducting a sensitivity analysis on allowed temperature deviation of the zinc together with a qualitative comparison of FCR and mFRR.


\subsection{Investments should facilitate power flexibility}

New investments into equipment and improvements in the industry process can be tailored to facilitate power flexibility. For example, instead of relying on mechanical relay switches for power consumption, frequency transformers or thyristors can be used instead for a modest marginal extra cost. The pay-back time for such an investment should ideally be within a few years, and other benefits could potentially be exploited as well, e.g., better feed forward planning.

\subsection{Value streams from power flexibility: FCR and mFRR}

Power consumption can potentially be flexible if the underlying industry process has a  degree of freedom in its operation. For example, power consumption to heat up zinc can be controlled as long as the temperature of the zinc is within some pre-specified thresholds that determines the degrees of freedom. For the zinc furnace, the lower temperature threshold specify solidification of molten zinc which must not happen as it can crack the furnace itself.

The ability to deviate from the operational baseline power consumption consumption can be monetized by participating in ancillary services such as FCR and mFRR. Furthermore, if these services are normally delivered by fossil fuel power plants, these are partly displaced and a CO$_{2}$ emission reduction is realized.

% In our work, we look only at monetary value and feasibility from FCR and mFRR as it is not yet clear how the CO$_{2}$ emissions impact can be quantified.

\subsection{Literature review}

Many studies have investigated demand-side flexibility. While most have focused on flexibility from residential households in an aggregated portfolio, some have also looked at industry processes with a significant electric energy consumption. None, though, specifically look at single-state industrial processes with in a real-life setting.

\IEEEpubidadjcol

Of these, freezers have been investigated thoroughly both with respect to mathematical representation \cite{shafiei2013modeling}, \cite{pedersen2016improving}  and as a source of flexibility \cite{sossan2016grey}, \cite{o2013modelling}, \cite{de2019leveraging}, \cite{misaghian2022fast}, \cite{vrettos2016fast}. Freezers are potentially a great source of flexibility because of the thermal inertia in the frozen food. However, supermarket freezers are complicated systems where individual freezer displays are dependent as they are fed cold air from the same compressor rack. Furthermore, there are many food regulations that complicates control of freezers with respect to allowed temperature deviations.

Much work has also gone into understanding heat dynamics of medium-large buildings \cite{thilker2021non} in order to operate them intelligently with respect to electricity prices and the grid \cite{contreras2018tractable}, \cite{finck2018quantifying}. However, buildings are fairly complicated in terms of control with potentially many different (dependent) heterogeneous assets such as ventilation units, heat pumps, water tanks, and heating units. Many single-state industrial processes are perhaps better suited for flexibility provision as they are more simple and easily modifiable.

In \cite{junker2020stochastic}, a water tower is characterized using stochastic differential equations for best planning how water should be pumped in relation to spot prices. Although not specifically for ancillary services, this serves as an illustrative example how operational control can benefit from knowledge of electricity prices in a fairly simple industry process.

The most simple mathematical model of a thermostatically controlled load (TCL), was presented in Hao et. al. \cite{hao2014aggregate}. Here, a first-order model with two terms fully characterized the TCL: a term that explains temperature losses due to temperature differences and a term that explains temperature gains due to a power source. We expand upon this model in this paper, using instead a 4th order model to characterize a zinc furnace.

While a simple, linear model is used for modelling FCR provision in this paper, the model introduced in (cite own paper) is used to model mFRR provision.



\subsection{Our contribution}

For a realistic energy intensive industry process using real data, we investigate if there is any incentive to provide power flexibility or to invest in equipment for the enablement of flexibility provision.

Specifically, we investigate a real-life industry process using actual (anonymized) data collected from 2022. We describe the process in detail and show how it can be adapted to cater for power flexibility by switching from ON/OFF power control to continuous power control.

The monetary value from the power flexibility is shown for FCR and mFRR provision in DK1\footnote{DK1 is the western part of Denmark and shares the same frequency as continental Europe} in Denmark. We provide an upper bound of flexibility earnings using a full hindsight optimization of each service and its impact on the temperature, i.e., the state of the industry process, is analyzed. We discuss and show how the allowed degrees of freedom in the temperature deviation impact the monetary value of flexibility provision.

\textbf{Should we instead provide paper structure here??}

\section{Adapting an industry process to provide flexibility}\label{sec:zinc_furnace_description}

In this section, the galvanizing process is described in detail along with its ability to provide power flexibility by converting to a continuous control mechanism.

We first describe how the temperature of the molten zinc can be characterized using state-space models. Then it is shown how this model can be used for subsequent simulation of continuous control. Afterwards, it it shown how power flexibility can be monetized in FCR and mFRR.

\subsection{Characterizing a zinc furnace as a TCL}



\subsubsection{Description of zinc furnace in a galvanizing process}

The industry process exemplified in this paper is a galvanization process, in which the authors have kindly been allowed to see and learn about by DOT Nordic. Steel elements are galvanized by lowering them into a molten zinc furnace at around 450 C$^{\circ}$. As seen in Figure \ref{fig:furnace_schematic_tikz}, the furnace is heated up by resistive elements placed on the sides of the furnace (in an inner cavity). The upper and lower zones of the furnace are controlled separately with $P^{\text{u}}$ and $P^{\text{l}}$ representing the power supplied to the upper and lower zone, respectively. Temperature sensors are placed in each zone at the end of the furnace on the wall as denoted by $T^{\text{wu}}$ and $T^{\text{wl}}$, hence the actual zinc temperatures, $(T^{\text{zu}})$ and $(T^{\text{zl}})$, are latent, unobserved states. The lid of the furnace is removed when lowering steel into the furnace.

\begin{figure}[!t]
    \centering
    \includestandalone[width=0.5\columnwidth]{figures/furnace_schematic_tikz}
    \caption{Schematic of the zinc furnace. Resistive elements are places on both sides of the furnace in inner cavities for both zones. The power consumptions of the resistive elements are denoted $P^{\text{u}}$ and $P^{\text{l}}$. At the end of the furnace, two temperature sensors are located, essentially measuring the wall temperatures, $T^{\text{wu}}$ and $T^{\text{wl}}$. Hence, the actual zinc temperatures are unobserved. The lid is off during the galvanizing process.}
    \label{fig:furnace_schematic_tikz}
\end{figure}

Figure \ref{fig:data_visualization} shows the anonymized\footnote{Only the scale of the temperature has been anonymized.} one-minute resolution data provided by DOT Nordic. The upper panel shows the power consumption of the lower and upper zones, respectively, and the total consumption as well. It is clearly seen how two regimes are immediately identified: one where the lid is on the furnace and the power consumption is quite low, and one where the lid is off corresponding to a high power consumption due to direct temperature losses to the ambient. In both the middle and bottom panel of Figure \ref{fig:data_visualization}, the temperatures behave differently in either regime: when the lid is on, the temperature varies slowly while it varies rapidly when the the lid is off. Furthermore, the temperature dynamics are generally slower in the lower zone of the furnace (bottom panel), and its temperature setpoint is also a bit lower. The ON/OFF control of the power consumption to the two zones are visualized by the state of four contactors - two for each zone - as seen in the middle and bottom panel. The control logic is simple for a given zone: one contactor is switched ON when the temperature goes belows a pre-specified threshold, and the other one is turned ON as well if the temperature declines further below another pre-specified threshold. Exactly the same logic applies when the temperature rises above two pre-specified upper thresholds. This logic is statically programmed for both zones independently.

\begin{figure}[!t]
    \centering
    \includegraphics[width=\columnwidth]{figures/data_visualization.png}
    \caption{\textbf{Top}: Power consumption for lower and upper zone as well as total consumption. \textbf{Middle}: Temperature and contactor switches in upper zone. \textbf{Bottom}: Temperature and contactor switches in lower zone.}
    \label{fig:data_visualization}
\end{figure}

\subsubsection{Thermal modelling of zinc furnace}

To model the temperature dynamics in both zones, a fourth-order state space model is used (\ref{eq:StateSpaceModel}):

\begin{subequations}\label{eq:StateSpaceModel}
    \begin{align}
        T^{\text{zu}}_{t+1} & = T^{\text{zu}}_{t} + dt \cdot \frac{1}{C^{\text{zu}}}\Bigl( \frac{1}{R^{\text{zuzl}}} (T^{\text{zl}}_{t} - T^{\text{zu}}_{t}) \notag                     \\ & \mspace{5mu} + \frac{1}{R^{\text{wz}}} (T^{\text{wu}}_{t} - T^{\text{zu}}_{t}) \Bigr) \label{eq1:StateSpaceModel} \\
        T^{\text{zl}}_{t+1} & = T^{\text{zl}}_{t} + dt \cdot \frac{1}{C^{\text{zl}}}\Bigl( \frac{1}{R^{\text{zuzl}}} (T^{\text{zu}}_{t} - T^{\text{zl}}_{t}) \notag                     \\ & \mspace{5mu} + \frac{1}{R^{\text{wz}}} (T^{\text{wl}}_{t} - T^{\text{zl}}_{t}) \Bigr) \label{eq2:StateSpaceModel} \\
        T^{\text{wu}}_{t+1} & = T^{\text{wu}}_{t} + dt \cdot \frac{1}{C^{\text{wu}}}\Bigl( (1-\mathbbm{1}^{\text{lid}}) \frac{1}{R^{\text{wua},1}} (T^{a} - T^{\text{wu}}_{t}) \notag + \\ & \mspace{5mu} \mathbbm{1}^{\text{lid}} \frac{1}{R^{\text{wua},2}} (T^{\text{a}} - T^{\text{wu}}_{t}) + \frac{1}{R^{\text{ww}}} (T^{\text{wl}}_{t} - T^{\text{wu}}_{t}) \notag \\ & \mspace{5mu} + \frac{1}{R^{\text{wz}}} (T^{\text{zu}}_{t} - T^{\text{wu}}_{t}) + p^{\text{u}}_{t} \Bigr) \label{eq3:StateSpaceModel} \\
        T^{\text{wl}}_{t+1} & = T^{\text{wl}}_{t} + dt \cdot \frac{1}{C^{\text{wl}}}\Bigl( \frac{1}{R^{\text{wla}}} (T^{\text{a}} - T^{\text{wl}}_{t}) \notag                           \\ & \mspace{5mu} + \frac{1}{R^{\text{ww}}} (T^{\text{wu}}_{t} - T^{\text{wl}}_{t}) + \frac{1}{R^{\text{wz}}} (T^{\text{zl}}_{t} - T^{\text{wl}}_{t}) + p^{\text{l}}_{t} \Bigr) \label{eq4:StateSpaceModel}
    \end{align}
\end{subequations}

Here, (\ref{eq1:StateSpaceModel}) and (\ref{eq3:StateSpaceModel}) represents the temperature of the zinc in the upper and lower zone, respectively. For both, there is a heat exchange between the furnace walls and the other zinc zone. In (\ref{eq3:StateSpaceModel}), the temperature dynamics of the upper part of the wall of the zinc furnace is shown. It depends on the heat loss to the ambient temperature, $T^{\text{a}}$, heat exchange with the lower wall, $T^{\text{wl}}$, with the zinc in the upper zone, $T^{\text{zu}}$, and the heat added from the resistive elements in the upper zone, $p^{\text{u}}$. Furthermore, there are two different resistance coefficients, $R^{\text{wua},1}$ and $R^{\text{wua},2}$, depending on whether the lid is on ($ \mathbbm{1}^{\text{lid}} = 1$) or off ($ \mathbbm{1}^{\text{lid}} = 0$). The index $t$, represents a given minute.

The parameters and latent states in (\ref{eq:StateSpaceModel}) have been estimated using CTSM-R \cite{juhl2016ctsmr}. The one-step residuals in the estimation procedure are shown in Figure \ref{fig:4thOrderModelValidation} and resembles white noise (middle panel). Furthermore, the autocorrelation shows no significant lags, and the cumulated periodogram shows no frequencies with significant power. Hence, the model in (\ref{eq:StateSpaceModel}) captures the temperature dynamics well for both the upper and lower zone of the furnace.


\begin{figure}[!t]
    \centering
    \includegraphics[width=\columnwidth]{figures/4thOrderModelValidation.png}
    \caption{Validation of state-space model in (\ref{eq:StateSpaceModel}). \textbf{Top}: upper zone wall temperature. \textbf{Bottom}: lower zone wall temperature. \textbf{Left}: autocorrelations of the model residuals. \textbf{Middle}: residuals. \textrm{Right}: cumulated periodogram of the residuals.}
    \label{fig:4thOrderModelValidation}
\end{figure}


\subsubsection{Going from ON/OFF control to steady-state control}

As seen in Figure \ref{fig:data_visualization}, the power consumption is rather unpredictable and varying with the ON/OFF control mechanism. Hence, it is difficult to actually know the operational baseline power consumption of the zinc furnace. Therefore, if the furnace should benefit from providing flexibility to the power grid through ancillary services, it is necessary to change the control structure to a more granular one.\footnote{In theory, the furnace can still participate in the current market structure of mFRR with 1-hour blocks if its part of a bigger portfolio that can compensate for its ON/OFF behavior.}

Using the model in (\ref{eq:StateSpaceModel}), we can estimate the steady-state power consumption for both regimes, i.e., when the lid is on and off:

\begin{subequations}\label{eq:steady-state-power}
    \begin{align}
        p^{\text{u},\text{SS},1} & = \frac{T^{\text{u},\text{sp}} - T^{\text{a}}}{R^{\text{wua},1}}  + \frac{T^{\text{u},\text{sp}}-T^{\text{l},\text{sp}}}{R^{\text{ww}}} \label{eq1:steady-state-power} \\
        p^{\text{u},\text{SS},2} & = \frac{T^{\text{u},\text{sp}} - T^{\text{a}}}{R^{\text{wua},2}} + \frac{T^{\text{u},\text{sp}}-T^{\text{l},\text{sp}}}{R^{\text{ww}}} \label{eq2:steady-state-power}  \\
        p^{\text{l},\text{SS}}   & = \frac{T^{\text{l},\text{sp}} - T^{\text{a}}}{R^{\text{wla}}} - \frac{T^{\text{u},\text{sp}}-T^{\text{l},\text{sp}}}{R^{\text{ww}}} \label{eq3:steady-state-power}
    \end{align}
\end{subequations}

Here, $T^{\text{l},\text{sp}}$ and $T^{\text{u},\text{sp}}$ are the pre-specified setpoints in the lower and upper zone, respectively, and $p^{\text{u},\text{SS},1}$ and $p^{\text{u},\text{SS},2}$ are the steady-state power consumption when the lid is OFF and ON, respectively, for the upper zone. Finally, $p^{\text{l},\text{SS}}$ is the steady-state power consumption for the lower zone.

Figure \ref{fig:4thOrderModelVisualizationSteadyState} shows a 24-hour simulation (1440 time steps) of (\ref{eq:StateSpaceModel}) which was chosen to include both regimes where the lid either on or off. The original data is shown in dashed, black lines and it can clearly be seen how the steady-state power consumptions (\ref{eq:steady-state-power}) and temperatures are much more stable and predictable. To achieve this, a controller would need be needed to keep the temperature at the setpoint. The purpose of the steady-state operation is to highlight the benefit of flexibility provision from a more predictable, operational baseline consumption.

\begin{figure}[!t]
    \centering
    \includegraphics[width=\columnwidth]{figures/4thOrderModelVisualizationSteadyState.png}
    \caption{Simulation of (\ref{eq:StateSpaceModel}) with $p^{\text{u}}$ and $p^{\text{l}}$ set to the steady-state consumptions as specified in (\ref{eq:steady-state-power}). \textbf{Top}: total power consumption in steady-state and original data (with ON/OFF control). \textbf{Middle}: upper zone wall temperature at steady-state power consumption and original data (with ON/OFF control). \textbf{Bottom}: lower zone wall temperature at steady-state power consumption and original data (with ON/OFF control).}
    \label{fig:4thOrderModelVisualizationSteadyState}
\end{figure}

Furthermore, other benefits of using a more continuous power control include more optimal use of the heat in the furnace by being able to integrate the dipping schedule to the power consumption. Hence, smarter feed forward planning of the lid can be achieved and made possible by utilizing a smarter control of the power sources.

\subsection{FCR}

FCR is the fastest responding ancillary service in DK1 in Denmark. Power must be adjusted according to frequency deviations around $\pm 200$ mHz from 50 Hz. The service is normally delivery by thermal power plants and batteries, but flexible demand can potentially also deliver FCR; either by control of individual assets or from an aggregation of multiple assets that can be turned ON/OFF in a manner that adheres to the frequency response requirements.

The service is bought by the Danish TSO in 4-hour blocks in one auction for the next day. The auction starts at 8 AM the day before delivery. Flexibility providers receive the marginal price of the auction as payment for providing capacity.

Figure \ref{fig:fcr_prices_2022} shows the FCR and spot prices in DK1 for all of 2022. Notice how prices went down after September where the Danish TSO opened up market participation from all of continental Europe. Furthermore, FCR prices have been somewhat proportional to spot prices as well.

\begin{figure}[!t]
    \centering
    \includegraphics[width=\columnwidth]{figures/fcr_prices.png}
    \caption{FCR and spot prices for 2022 in DK1, Denmark. In September, FCR tenders included continental Europe.}
    \label{fig:fcr_prices_2022}
\end{figure}

\subsection{mFRR}

mFRR is an ancillary service operated by the TSO. It is the last service deployed after a frequency drop, and the energy content can be several megawatts. In DK1 in Denmark, the TSO procures around 600 megawatts of mFRR (cite).

As described in (cite own paper), procurement of mFRR reserves happens before the day-ahead market clearing, and the subsequent balancing bid happens after the day-ahead market clearing. mFRR providers are paid at the marginal price according to the capacity they offer. In real-time, activation of reserves happens when the TSO demands it which leads to a balancing price higher than the spot price. mFRR providers are paid according to the amount they activate and penalized if they do not deliver their promised capacity. For flexible demand with rebound, such as a zinc furnace, there will be an inevitable rebound which requires additional energy.

Figure \ref{fig:mfrr_prices_2022} shows the distribution of balancing minus spot prices (also called balance price differentials) ordered from low to high in red. Corresponding capacity prices are shown in blue. When balance price differentials are below zero, the power grid down-regulated, i.e., suppler was greater than demand. When price differentials are above zero, the power grid up-regulated, i.e., demand was higher than supply. mFRR capacity is for up-regulation but flexible providers can both down-regulate and up-regulate in the real-time balancing market if they choose. Here, however, we only consider the mFRR up-regulation capacity market where the zinc furnace is paid for both capacity (as shown in blue in Figure \ref{fig:mfrr_prices_2022}) and actual up-regulations (as shown in blue in Figure \ref{fig:mfrr_prices_2022} when balance price differential is above zero).

\begin{figure}[!t]
    \centering
    \includegraphics[width=\columnwidth]{figures/mfrr_prices.png}
    \caption{mFRR reserve prices prices and balancing price differentials in 2022 in ascending order.}
    \label{fig:mfrr_prices_2022}
\end{figure}

\section{Optimization model}

The optimization models for FCR and mFRR are presented in this section. First, the FCR model is presented. Second, the mFRR model is presented in compact form with the full model shown in (cite own paper). Both models are deterministic with full hindsight on prices. We thus provide an upper bound in the flexibility potential which is useful to know when deciding to make investments that enables power flexibility provision. For example, if the upper bound potential is not attractive, then investments should not be made.

\subsection{FCR}

The linear optimization model for FCR reserve bidding and activation is shown in compact form in (\ref{P1:compact_model_fcr}) with bold indicating time vectors. The objective in (\ref{P1:eq1}) maximizes revenue from capacity payments and minimizes penalties. The aggregator incurs a penalty cost of at $\bm{\lambda}^{\text{Pen}}$ DKK/kWh whenever the reserve, $\bm{p}^{r}$ cannot be met.

Variables related to bidding capacity and penalty are indexed by hours, $\bm{\Gamma}^{h}$, and variables related to real-time control are indexed by minute, $\bm{\Gamma}^{t}$.

% # TODO: be aware of different time scale: minute vs hourly!!
\begin{subequations}\label{P1:compact_model_fcr}
    \begin{align}
        \underset{\bm{\Gamma}^{h},\bm{\Gamma}^{t}}{\textrm{max}} \quad & \bm{p}^{r} \bm{\lambda}^{\text{FCR}} - \bm{s}\bm{\lambda}^{\text{Pen}} \label{P1:eq1}
        \\
        s.t \quad                                                      & q(\bm{\Gamma}^{h}, ) \leq 0 \label{P1:eq2}                                                                                                                                                  \\
        \quad                                                          & \text{State-space model in } (\ref{eq:StateSpaceModel}) \label{P1:eq8}                                                                                                                      \\
        \quad
                                                                       & \bm{\Gamma}^{h} = \Bigl( \bm{p}^{r}, \bm{s} \Bigr) \in \mathbb{R}^{n}  \label{P1:eq9}                                                                                                       \\
        \quad
                                                                       & \bm{\Gamma}^{t} = \Bigl( \bm{p}, \bm{p}_{z}, \bm{s}^{\prime}_{z}, \bm{T}^{\text{zu}}, \bm{T}^{\text{zl}}, \bm{T}^{\text{wu}}, \bm{T}^{\text{wl}} \Bigr) \in \mathbb{R}^{n}  \label{P1:eq10}
    \end{align}
\end{subequations}

The inequality constraints in (\ref{P1:eq2}) are shown in \ref{P1:constraints}. Since FCR is a symmetric service, auxilliary variables $\bm{s}^{\prime}$ are introduced to account for the power not delivered as shown in (\ref{P1:co3})-(\ref{P1:co4}). The index $z$ denotes the two zones in the zinc furnace, and the total reserve from the zinc furnace is simply the sum of these as seen in (\ref{P1:co6})-(\ref{P1:co7}).

\begin{subequations} \label{P1:constraints}
    \begin{align}
              & \bm{p}_{z} = \bm{F} \bm{p}^{r}_{z} + \bm{s}^{\prime}_{z} + \bm{P}^{\text{Base}}_{z}, \quad \forall{z}  \label{P1:co2} \\
        \quad & \bm{s}_{z} \geq \bm{s}^{\prime}_{z}, \quad \forall{z}  \label{P1:co3}
        \\
        \quad & \bm{s}_{z} \geq -\bm{s}^{\prime}_{z}, \quad \forall{z}  \label{P1:co4}
        \\
        \quad & \bm{p}^{r}_{z} \leq \bm{P}_{z}^{\text{Base}} , \quad \forall{z} \label{P1:co8}
        \\
        \quad & \bm{p} = \sum_{z} \bm{p}_{z}   \label{P1:co5}                                                                         \\
        \quad & \bm{p}^{r} = \sum_{z} \bm{p}^{r}_{z}   \label{P1:co6}                                                                 \\
        \quad & \bm{s} = \frac{1}{60} \sum_{z} \bm{s}_{z}   \label{P1:co7}
    \end{align}
\end{subequations}

The parameter $\bm{F}$ is normalized between -1 and 1 depending on the frequency such that it represents the normalized response required:

\begin{equation}
    F_{t} =
    \begin{cases}
        -1,                             & \text{if}\ F_{t} \leq 49.8\ \text{mHz}                                           \\
        \frac{F_{t}-50+0.02}{0.2-0.02}, & \text{if}\ F_{t} \leq 49.98\ \text{mHz}\ \text{and}\ F_{t} \geq 49.8\ \text{mHz} \\
        \frac{F_{t}-50-0.02}{0.2-0.02}, & \text{if}\ F_{t} \leq 50.2\ \text{mHz}\ \text{and}\ F_{t} \geq 50.02\ \text{mHz} \\
        1,                              & \text{if}\ F_{t} \geq 50.2 \text{mHz}                                            \\
    \end{cases}
\end{equation}

\subsection{mFRR}

The optimization model for mFRR is identical to (cite own paper) with no scenarios. The objective maximizes first-stage decisions, i.e., payments for reserve capacity, by finding optimal reserve quantities and bids, $\bm{p}^{r,\uparrow}$ and $\bm{\lambda}^{\text{bid}}$, respectively. Second-stage decisions are merely a consequence of first-stage decisions and are related to real-time operation with cost components of activation, rebound, and penalty terms. See (cite own paper) for a detailed description of the model.

\begin{subequations}\label{P2:compact_model_mfrr}
    \begin{align}
        \underset{\bm{p}^{r,\uparrow}, \bm{\lambda}^{\text{bid}}, \bm{\Gamma}}{\textrm{max}} \quad & f(\bm{p}^{r,\uparrow}) + g(\bm{\Gamma}) \label{P2:eq1}
        \\
        s.t \quad                                                                                  & h(\bm{p}^{r,\uparrow}, \bm{\lambda}^{\text{bid}}, \bm{\Gamma}) \leq 0\label{P2:eq2}                                             \\
        \quad                                                                                      & \text{State-space model in } (\ref{eq:StateSpaceModel}) \label{P2:eq3}
        \\
        \quad                                                                                      & \bm{\Gamma} = \Bigl( \bm{p}^{r,\uparrow}, \bm{\lambda}^{\text{bid}}, \bm{p}^{b,\uparrow}, \bm{p}^{b,\downarrow}, \bm{s}, \notag \\ \quad & \quad \bm{T}^{z, B}, \bm{T}^{z}, \bm{\phi}, \bm{g} \Bigr) \in \mathbb{R}^{n}  \label{P2:eq4}
        \\
        \quad                                                                                      & \bm{u}, \bm{z}, \bm{y} \in \{0,1\}  \label{P2:eq5}
    \end{align}
\end{subequations}


\section{Results and discussion}

The results have the following structure: First, we discuss monetary savings for both FCR and mFRR. Second, we investigate worst-case days (in terms of total up-regulation) in 2022 for both FCR in mFRR. This includes explanations of the zinc furnace's their behavior in those days. Third, the zinc temperature's impact on monetary savings is presented and analyzed.

Figure \ref{fig:cumulative_cost_comparison} shows the cumulative operational costs in 2022 for the zinc furnace participating in FCR, mFRR, and state of today as a reference. Clearly, FCR provided significant revenue during the summer when prices were high. Revenues for FCR compared to mFRR seem to be mostly due to this as FCR stagnates a bit after September (where FCR bidding was open to all of Europe and not just Western Denmark).

Keeping in mind that Figure \ref{fig:cumulative_cost_comparison} shows the upper bound potential, there is still significant benefit for a zinc furnace to participate with its flexibility. An annual saving of around 0.5 mil. DKK can quickly recover investment costs needed to enable smart control of the power supply to the furnace.\footnote{According to the owner of the zinc furnace, investment costs for thyristors are around 0.5 mil. DKK.}


\begin{figure}[!t]
    \centering
    \includegraphics[width=\columnwidth]{figures/cumulative_cost_comparison.png}
    \caption{Cumulative operational cost for 2022 when participating in FCR and mFRR compared to current baseline operational costs.}
    \label{fig:cumulative_cost_comparison}
\end{figure}

To investigate how flexibility provision differs for FCR and mFRR, the two worst days in 2022 are extracted and analyzed.

For FCR, the worst day is defined as the day where the frequency deviated the most from 50 Hz. This day is shown in Figure \ref{fig:fcr_single_case}. First of all, it can clearly be seen how all of the operational baseline consumption is bid as reserve (top plot). This happens in all days during the year and maximizes the revenue from FCR. The implication is that the furnace has to provide a frequency response \textit{continuously} for all hours all year long which has an impact on the temperature (middle upper plot). However, the temperature only declines slightly and is still well within the thresholds before solidification occurs. The frequency response (middle lower plot) shows how most of the response is up-regulation, i.e., turning down the power consumption which happens when the frequency in the grid is below 50 Hz.

\begin{figure}[!t]
    \centering
    \includegraphics[width=\columnwidth]{figures/fcr_single_case.png}
    \caption{Worst-case day in 2022 in terms of cumulative frequency deviation to 50 Hz. \textbf{Top}: total baseline operational power and reserve capacity. \textbf{Middle upper}: wall temperature of upper and lower zone of the furnace with and without (denoted "Base") frequency participation. \textbf{Middle lower}: operational power consumption of lower and upper zone as well as total consumption with the baseline operational consumption in solid lines. \textbf{Bottom}: spot and FCR prices for worst-case day.}
    \label{fig:fcr_single_case}
\end{figure}

For mFRR, the reserve quantities are slightly lower than for FCR as seen in Figure \ref{fig:mfrr_single_case} (top). This is due the penalty of not being able to deliver all up-regulation that was promised as can be seen by comparing the bottom plot with the reserve (top) and actual consumption (middle lower): whenever the bid is lower than the balancing price \textit{and} a reserve is promised, then the actual power consumption should correspond to $P^{\text{Base}}_{h} - p_{h}^{r, \uparrow}$. For example, this does not happen in hour 12. Also, the optimization model in (\ref{P2:compact_model_mfrr}) requires an immediate rebound after activation which prohibits the zinc furnace from up-regulating more than 12 hours per day. For these reasons, the reserved power is not equal to the operational baseline power.

Furthermore, turning off the power consumption for five consecutive hours severally affect the temperature as seen in the middle, upper plot. Here, the impact is much bigger than for FCR and exceeds the pre-specified thresholds by an order of magnitude.

\begin{figure}[!t]
    \centering
    \includegraphics[width=\columnwidth]{figures/mfrr_single_case.png}
    \caption{Worst-case day in 2022 in terms of up-regulation in the power grid. \textbf{Top}: total baseline operational power and mFRR reserve capacity. \textbf{Middle upper}: wall temperature of upper and lower zone of the furnace with and without (denoted "Base") frequency participation. \textbf{Middle lower}: operational power consumption of lower and upper zone as well as total consumption with the baseline operational consumption in solid lines. \textbf{Bottom}: spot and balancing prices for worst-case day together with the mFRR activation bid.}
    \label{fig:mfrr_single_case}
\end{figure}

To assess the available flexibility in the zinc furnace, it is prudent to investigate the allowed temperature deviation's impact on the monetary savings for both FCR and mFRR. As indicated already in Figure \ref{fig:fcr_single_case}, the temperature is not affected to any significant degree for FCR. This is again shown in Figure \ref{fig:profit_vs_delta_temp_fcr} where an allowed temperature deviation of 1 C$^{\circ}$ is enough to provide substantial monetary savings.

\begin{figure}[!t]
    \centering
    \includegraphics[width=\columnwidth]{figures/profit_vs_delta_temp_fcr.png}
    \caption{Monetary savings when delivering FCR as a function of constraining the allowed temperature deviation of the zinc to the setpoint.}
    \label{fig:profit_vs_delta_temp_fcr}
\end{figure}

The same can not be said of mFRR as seen in Figure \ref{fig:profit_vs_delta_temp_mfrr} and alluded to previously in Figure \ref{fig:mfrr_single_case}. Figure \ref{fig:profit_vs_delta_temp_mfrr} clearly shows diminishing value at lower temperature deviations while the best savings are obtained when allowing temperature deviations of at least 6 $^{\circ}$.
% (which is of course unrealistic to cope with for the zinc furnace operator).

\begin{figure}[!t]
    \centering
    \includegraphics[width=\columnwidth]{figures/profit_vs_delta_temp_mfrr_and_energy.png}
    \caption{Monetary savings when delivering mFRR as a function of constraining the allowed temperature deviation of the zinc to the setpoint.}
    \label{fig:profit_vs_delta_temp_mfrr}
\end{figure}


One has to remember that the revenue generated from FCR and mFRR needs to be shared between the aggregator and flexible consumer as explained in \cite{gade2022ecosystem}. Hence, the provided upper bound savings reported are also bound to individual payment agreements with the aggregator. That can potentially make it less attractive for flexible demand to participate in ancillary services. However, FCR still seems like an obvious option and likewise for the investments made to enable FCR provision. mFRR, however, is not too attractive as it can have a detrimental impact on the zinc temperature on days with severe up-regulation. Other, perhaps less profitable, mFRR strategies must be employed. For example, only providing real-time balancing without reserves is a viable option since no commitments are made beforehand.

Other revenue streams, such as load shifting and aFRR, were not considered here, but they can potentially also be profitable in certain market regimes with high aFRR demand or very volatile spot prices.

\input{sections/conclusion}

This paper explored how a simple, single-state industry process can make modest investments to enable flexibility provision in FCR and mFRR. By switching static control logic to continuous power control, as exemplified by a real-world case study of a zinc galvanizing process, economic benefits can achieved. Comparing FCR and mFRR, it is certainly not realistic to expect a full up-regulation of five consecutive hours for mFRR which will not happen when delivering FCR. And FCR pays better as well. FCR provision is thus a very attractive opportunity for a zinc furnace owner as it provides a stable and passive source of income once investments into continuous power control are made.

% \input{sections/appendix}

\section*{Acknowledgement}

The authors would like to acknowledge the financial support from Innovation Fund Denmark under grant number 0153-00205B for partially funding the work in this paper. The authors would also like to thank Haris Ziras (DTU) for his feedback on the paper and his valuable input on the development of the grey-box model of the freezer. The authors would also like to thank Coop and AK-Centralen for providing the freezer and power data from a supermarket used in this paper.

% The authors would also like to thank Christian Ahn Albertsen (IBM Client Innovation Center) for all our discussions and his feedback which has been particularly valuable regarding practical challenges faced for demand-response. The authors would also like to thank Bennevis Crowley (DTU) for reading the paper and providing feedback on corrections, figures, and phrases. The authors would also like to thank the Centre for Utilities and Supply at Energistyrelsen (Danish Energy Agency) for allowing us to present our work to them while also elaborating their perspective on Market Model 3.0 and their intentions.



\bibliographystyle{IEEEtran}
% \bibliography{tex/bibliography/Bibliography}
\bibliography{bibliography/Bibliography}

% \newpage

% \begin{IEEEbiographynophoto}{Peter A.V. Gade}
%     is an Industrial PhD researcher at IBM and affiliated with the Technical University of Denmark, Kongens Lyngby, Denmark, in the Energy Markets and Analytics Section within the Power and Energy Systems division at the Wind and Energy Systems Department. His research focuses on demand-side flexibility and the revenue streams from utilization of demand-side flexibility. He holds a M.S. in Mathematical Modelling and Computing and a B.S. in Biomedical Engineering, both from the Technical University of Denmark.
% \end{IEEEbiographynophoto}
% \begin{IEEEbiographynophoto}{Trygve Skjøtskfit}
%     is an Associate Partner at IBM Denmark, with focus on energy transformation and demand-side flexibility. His solid experience and deep knowledge within intelligent energy systems, buildings, and civil infrastructures makes him a leading figure, strategic advisor, and a first mover in the flexibility market with a strong track record to find and deliver new cutting-edge solutions. He holds an MBA in Strategy from Universitat Pompeu Fabra, and a Master of Export Engineering from Copenhagen University, College of Engineering.
% \end{IEEEbiographynophoto}
% \begin{IEEEbiographynophoto}{Henrik W. Bindner}
%     received the MSc in Electrical Engineering from Technical University of Denmark in 1988. He is currently a senior researcher with the Department of Wind and Energy Systems, Technical University of Denmark. He is heading the \textit{Distributed Energy Systems} Section and his research interests include control and management of smart grids, active distribution networks, and integrated energy systems.
% \end{IEEEbiographynophoto}
% \begin{IEEEbiographynophoto}{Charalampos Ziras}
%     ...
% \end{IEEEbiographynophoto}
% \begin{IEEEbiographynophoto}{Jalal Kazempour}
%     is an Associate Professor with the Department of Wind and Energy Systems, Technical University of Denmark, where he is heading the \textit{Energy Markets and Analytics} Section. He received the Ph.D. degree in Electrical Engineering from the University of Castilla-La Mancha, Ciudad Real, Spain, in 2013. His research interests include intersection of multiple fields, including power and energy systems, electricity markets, optimization, game theory, and machine learning.
% \end{IEEEbiographynophoto}


\vfill

\end{document}
